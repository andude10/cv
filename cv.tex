%-----------------------------------------------------------------------------------------------------------------------------------------------%
%	The MIT License (MIT)
%
%	Copyright (c) 2021 Jitin Nair
%
%	Permission is hereby granted, free of charge, to any person obtaining a copy
%	of this software and associated documentation files (the "Software"), to deal
%	in the Software without restriction, including without limitation the rights
%	to use, copy, modify, merge, publish, distribute, sublicense, and/or sell
%	copies of the Software, and to permit persons to whom the Software is
%	furnished to do so, subject to the following conditions:
%
%	THE SOFTWARE IS PROVIDED "AS IS", WITHOUT WARRANTY OF ANY KIND, EXPRESS OR
%	IMPLIED, INCLUDING BUT NOT LIMITED TO THE WARRANTIES OF MERCHANTABILITY,
%	FITNESS FOR A PARTICULAR PURPOSE AND NONINFRINGEMENT. IN NO EVENT SHALL THE
%	AUTHORS OR COPYRIGHT HOLDERS BE LIABLE FOR ANY CLAIM, DAMAGES OR OTHER
%	LIABILITY, WHETHER IN AN ACTION OF CONTRACT, TORT OR OTHERWISE, ARISING FROM,
%	OUT OF OR IN CONNECTION WITH THE SOFTWARE OR THE USE OR OTHER DEALINGS IN
%	THE SOFTWARE.
%
%
%-----------------------------------------------------------------------------------------------------------------------------------------------%

%----------------------------------------------------------------------------------------
%	DOCUMENT DEFINITION
%----------------------------------------------------------------------------------------

% article class because we want to fully customize the page and not use a cv template
\documentclass[a4paper,12pt]{article}

%----------------------------------------------------------------------------------------
%	FONT
%----------------------------------------------------------------------------------------

% % fontspec allows you to use TTF/OTF fonts directly
% \usepackage{fontspec}
% \defaultfontfeatures{Ligatures=TeX}

% % modified for ShareLaTeX use
% \setmainfont[
% SmallCapsFont = Fontin-SmallCaps.otf,
% BoldFont = Fontin-Bold.otf,
% ItalicFont = Fontin-Italic.otf
% ]
% {Fontin.otf}

%----------------------------------------------------------------------------------------
%	PACKAGES
%----------------------------------------------------------------------------------------
\usepackage{url}
\usepackage{parskip}

%other packages for formatting
\RequirePackage{color}
\RequirePackage{graphicx}
\usepackage[usenames,dvipsnames]{xcolor}
\usepackage[scale=0.9]{geometry}

%tabularx environment
\usepackage{tabularx}

%for lists within experience section
\usepackage{enumitem}

% centered version of 'X' col. type
\newcolumntype{C}{>{\centering\arraybackslash}X}

%to prevent spillover of tabular into next pages
\usepackage{supertabular}
\usepackage{tabularx}
\newlength{\fullcollw}
\setlength{\fullcollw}{0.47\textwidth}

%custom \section
\usepackage{titlesec}
\usepackage{multicol}
\usepackage{multirow}

%CV Sections inspired by:
%http://stefano.italians.nl/archives/26
\titleformat{\section}{\Large\scshape\raggedright}{}{0em}{}[\titlerule]
\titlespacing{\section}{0pt}{10pt}{10pt}

%for publications
\usepackage[style=authoryear,sorting=ynt, maxbibnames=2]{biblatex}

%Setup hyperref package, and colours for links
\usepackage[unicode, draft=false]{hyperref}
\definecolor{linkcolour}{rgb}{0,0.2,0.6}
\hypersetup{colorlinks,breaklinks,urlcolor=linkcolour,linkcolor=linkcolour}
\addbibresource{citations.bib}
\setlength\bibitemsep{1em}

%for social icons
\usepackage{fontawesome5}

%debug page outer frames
%\usepackage{showframe}

%----------------------------------------------------------------------------------------
%	BEGIN DOCUMENT
%----------------------------------------------------------------------------------------
\begin{document}

% non-numbered pages
\pagestyle{empty}

%----------------------------------------------------------------------------------------
%	TITLE
%----------------------------------------------------------------------------------------

% \begin{tabularx}{\linewidth}{ @{}X X@{} }
% \huge{Your Name}\vspace{2pt} & \hfill \emoji{incoming-envelope} email@email.com \\
% \raisebox{-0.05\height}\faGithub\ username \ | \
% \raisebox{-0.00\height}\faLinkedin\ username \ | \ \raisebox{-0.05\height}\faGlobe \ mysite.com  & \hfill \emoji{calling} number
% \end{tabularx}

\begin{tabularx}{\linewidth}{@{} C @{}}
\Huge{Kotov Semyon} \\[7.5pt]
\href{https://github.com/andude10}{\raisebox{-0.05\height}\faGithub\ Github} \ $|$ \
\href{https://www.linkedin.com/in/semyon-kotof-aa6045251/}{\raisebox{-0.05\height}\faLinkedin\ LinkedIn} \ $|$ \
\href{https://andude10.github.io/}{\raisebox{-0.05\height}\faGlobe \ Personal website } \ $|$ \
\href{mailto:formalkotov@gmail.com}{\raisebox{-0.05\height}\faEnvelope \ formalkotov@gmail.com } \ $|$ \
\href{tel:+36205350225}{\raisebox{-0.05\height}\faMobile \ +36 20 535 0225} \\
\end{tabularx}

%----------------------------------------------------------------------------------------
%	EDUCATION
%----------------------------------------------------------------------------------------
\section{Education}
\begin{tabularx}{\linewidth}{@{}l X@{}}
\textbf{Eötvös Loránd University (ELTE)} & \hfill \normalsize Budapest, Hungray \\
BSc in Computer Science  & \hfill \normalsize \normalsize 2023 - 2026 \\

\end{tabularx}

%----------------------------------------------------------------------------------------
%	SKILLS
%----------------------------------------------------------------------------------------
\section{Skills}
\begin{tabularx}{\linewidth}{@{}l X@{}}
Languages &  \normalsize{C\# (.NET), Python, C  }\\
Technologies &  \normalsize{HTML/CSS, ASP.NET Core, SQL Server, Git, SQLite, SQL, Docker, WPF, EF Core  }\\
Other &  \normalsize{ OOP, FP, REST API, Testing, Design patterns }\\
\end{tabularx}


% ----------------------------------------------------------------------------------------
% EXPERIENCE SECTIONS
% ----------------------------------------------------------------------------------------

% %Experience
% \section{Work Experience}

% \begin{tabularx}{\linewidth}{ @{}l r@{} }
% \textbf{Designation} & \hfill Jan 2021 - present \\[3.75pt]
% \multicolumn{2}{@{}X@{}}{long long line of blah blah that will wrap when the table fills the column width long long line of blah blah that will wrap when the table fills the column width long long line of blah blah that will wrap when the table fills the column width long long line of blah blah that will wrap when the table fills the column width}  \\
% \end{tabularx}

% \begin{tabularx}{\linewidth}{ @{}l r@{} }
% \textbf{Designation} & \hfill Mar 2019 - Jan 2021 \\[3.75pt]
% \multicolumn{2}{@{}X@{}}{
% \begin{minipage}[t]{\linewidth}
%     \begin{itemize}[nosep,after=\strut, leftmargin=1em, itemsep=3pt]
%         \item[--] long long line of blah blah that will wrap when the table fills the column width
%         \item[--] again, long long line of blah blah that will wrap when the table fills the column width but this time even more long long line of blah blah. again, long long line of blah blah that will wrap when the table fills the column width but this time even more long long line of blah blah
%     \end{itemize}
%     \end{minipage}
% }
% \end{tabularx}

%Projects
\section{Projects}

\begin{tabularx}{\linewidth}{ @{}l r@{} }
\textbf{ Analysis tool for the data logger (2023)} | \emph{C\#} & \hfill \href{https://andude10.github.io/projects/Librotech Inspection}{Blog post} \\[3.75pt]
\multicolumn{2}{@{}X@{}}{
\begin{minipage}[t]{\linewidth}
    \begin{itemize}[nosep,after=\strut, leftmargin=1em, itemsep=3pt]
        \item[--] Offline cross-platform solution to analyze records collected by the company's data logger.
        \item[--] Faced poor graph performance with large amounts of data. Implemented the Douglas-Peucker algorithm, which reduced the number of points on the graph by 360\% without changing the graph view, resulting in a significant performance boost.
        \item[--] Developed page caching system to reduce RAM usage. When the user navigates to another page, the previous page is saved in a .json file that will be created back if needed.
        \item[--] To keep code clean and readable, I followed OOP, C\# code conventions and maintained documentation. Utilized MVVM architecture, Dependency Injection (DI), Fluent Interface Design and Singleton. Introduced logging using NLog to analyze bugs encountered by users. Introduced unit testing.
        \item[--] Found the causes of poor application performance using dotTrace, performance profiler.
        \item [--] Developed a minimalistic UI as per the user's request. Used asynchronous programming and the ReactiveUI library to make the interface responsive and smooth.
    \end{itemize}
    \end{minipage}
}  \\
\\
\textbf{Client for music service (2022)} | \emph{C\#, API, TPL} & \hfill \href{https://github.com/andude10/Avayandex-Music}{Source code} \\[3.75pt]
\multicolumn{2}{@{}X@{}}{
\begin{minipage}[t]{\linewidth}
    \begin{itemize}[nosep,after=\strut, leftmargin=1em, itemsep=3pt]
        \item[--] Cross-platform community client for popular online music service.
        \item[--] Implemented authorization, downloading/playing tracks, track queue, home page, user data and more.
        \item [--] Faced poor user experience due to slow app performance. It was because of large number of API requests in single thread. Introduced task parallel library (TPL), making interface much more responsive.
        \item [--] Contributed to open-source C\# API library.
    \end{itemize}
    \end{minipage}
}  \\
\\
\textbf{Online bookstore (2021)} | \emph{C\#, ASP.NET Core, SQL Server, Entity Framework} & \hfill \href{https://github.com/andude10/The-Tome/}{Source code} \\[3.75pt]
\multicolumn{2}{@{}X@{}}{
\begin{minipage}[t]{\linewidth}
    \begin{itemize}[nosep,after=\strut, leftmargin=1em, itemsep=3pt]
        \item [--] Web application for a fictional bookstore with basic (CRUD) functionality.
        \item [--] Stored data in SQL Server, made database queries using ORM (Entity Framework).
    \end{itemize}
    \end{minipage}
}  \\
\\
\end{tabularx}

%Projects
\section{Languages}

\begin{tabularx}{\linewidth}{@{}l X@{}}
English  &  \normalsize{C1 (Obtained IELTS Academic Certificate)}\\
\end{tabularx}

\vfill
\center{\footnotesize Last updated: \today}

\end{document}
